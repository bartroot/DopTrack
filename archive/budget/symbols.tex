% !TEX root = ./doptrack.budget.tex

%TODO: try \ensuremath to all of these, instead of $...$

%% ======================== flexible olerloading of plural forms of acronyms ========================== %%

%\acro and \acrodef derivatives that also define a plural form with the same label
\newcommand*{\acrosympluralbase}[5]{
	\acro			{#1}			[#2]{#3}
	\acroplural		{#1}			[#4]{#5}
}
\newcommand*{\acrodefsympluralbase}[5]{
	\acrodef		{#1}	[#2]{#3}
	\acrodefplural	{#1}	[#4]{#5}
}

%% ======================== overloading 'at epoch i' forms of acronyms ========================== %%

%when in IncludeListWithAllSymbolForms, print the list of symbols with all symbols, including plurals and $_i$ forms
\ifdefined\IncludeListWithAllSymbolForms

%\acro and \acrodef derivatives cumulative with \acrosympluralbase that also define the suffixes $_i$ in the short form and "at epoch i" in the long form.
\newcommand*{\acrosymplural}[5]{
	\acrosympluralbase		{#1}	{#2}		{#3}				{#4}		{#5}				(\acsp{#1}   \aclp{#1})   {\sf \bf #1}
	\acrosympluralbase		{#1.i}	{#2$_i$}	{#3 at epoch $i$}	{#4$_i$}	{#5 at epoch $i$}	(\acsp{#1.i} \aclp{#1.i}) {\sf \bf #1.i}
}
\newcommand*{\acrodefsymplural}[5]{
	\acrosympluralbase		{#1}	{#2}		{#3}				{#4}		{#5}				(\acsp{#1}   \aclp{#1})   {\sf \bf #1}
	\acrosympluralbase		{#1.i}	{#2$_i$}	{#3 at epoch $i$}	{#4$_i$}	{#5 at epoch $i$}	(\acsp{#1.i} \aclp{#1.i}) {\sf \bf #1.i}
}
%\acro and \acrodef derivatives cumulative with \acrosympluralbase that also DON'T define the suffixes $_i$ in the short form and "at epoch i" in the long form.
\newcommand*{\acrosympluralni}[5]{
	\acrosympluralbase		{#1}	{#2}		{#3}				{#4}		{#5}				(\acsp{#1}   \aclp{#1})   {\sf \bf #1}
}
\newcommand*{\acrodefsympluralni}[5]{
	\acrosympluralbase		{#1}	{#2}		{#3}				{#4}		{#5}				(\acsp{#1}   \aclp{#1})   {\sf \bf #1}
}

%when not in IncludeListWithAllSymbolForms, print the list of symbols with only the top symbols and no plurals and $_i$ forms
\else

%\acro and \acrodef derivatives cumulative with \acrosympluralbase that also define the suffixes $_i$ in the short form and "at epoch i" in the long form.
\newcommand*{\acrosymplural}[5]{
	\acrosympluralbase		{#1}	{#2}		{#3}				{#4}		{#5}
	\acrodefsympluralbase	{#1.i}	{#2$_i$}	{#3 at epoch $i$}	{#4$_i$}	{#5 at epoch $i$}
}
\newcommand*{\acrodefsymplural}[5]{
	\acrodefsympluralbase	{#1}	{#2}		{#3}				{#4}		{#5}
	\acrodefsympluralbase	{#1.i}	{#2$_i$}	{#3 at epoch $i$}	{#4$_i$}	{#5 at epoch $i$}
}
%\acro and \acrodef derivatives cumulative with \acrosympluralbase that also DON'T define the suffixes $_i$ in the short form and "at epoch i" in the long form.
\newcommand*{\acrosympluralni}[5]{
	\acrosympluralbase		{#1}	{#2}		{#3}				{#4}		{#5}
}
\newcommand*{\acrodefsympluralni}[5]{
	\acrodefsympluralbase	{#1}	{#2}		{#3}				{#4}		{#5}
}

\fi

%% ======================== olerloading 'at epoch i' and plural forms of acronyms ========================== %%

%\acro and \acrodef derivatives cumulative with \acrosymplural that replicate the short and long forms, appending "s" after the long form
\newcommand*{\acrosym}			[3]{\acrosymplural		{#1}{#2}{#3}{#2}{#3s}}
\newcommand*{\acrodefsym}		[3]{\acrodefsymplural	{#1}{#2}{#3}{#2}{#3s}}

%\acro and \acrodef derivatives cumulative with \acrosympluralni that replicate the short and long forms, WITHOUT the "at epoch i" variation
\newcommand*{\acrosymni}		[3]{\acrosympluralni	{#1}{#2}{#3}{#2}{#3s}}
\newcommand*{\acrodefsymni}	[3]{\acrodefsympluralni	{#1}{#2}{#3}{#2}{#3s}}

%\acro and \acrodef derivatives cumulative with \acrosymplural that replicate the short and long forms, WITHOUT appending "s" after the long form
\newcommand*{\acrosymnp}		[3]{\acrosymplural		{#1}{#2}{#3}{#2}{#3}}
\newcommand*{\acrodefsymnp}	[3]{\acrodefsymplural	{#1}{#2}{#3}{#2}{#3}}

%\acro and \acrodef derivatives cumulative with \acrosympluralni that replicate the short and long forms, WITHOUT "at epoch i" variation and "s" after the long form
\newcommand*{\acrosymnip}		[3]{\acrosympluralni	{#1}{#2}{#3}{#2}{#3}}
\newcommand*{\acrodefsymnip}	[3]{\acrodefsympluralni	{#1}{#2}{#3}{#2}{#3}}


%\acro and \acrodef derivatives cumulative with \acrosym that creates a composite symbol and appends "s" after the long form
\newcommand*{\acrocomp}			[2]{\acrosym		{#1.#2}{\acs {#1}\acs{#2}}{\acl{#2} \acl {#1}}}
\newcommand*{\acrodefcomp}		[2]{\acrodefsym		{#1.#2}{\acs {#1}\acs{#2}}{\acl{#2} \acl {#1}}}

%\acro and \acrodef derivatives cumulative with \acrosymni that creates a composite symbol, WITHOUT the "at epoch i" variation
\newcommand*{\acrocompni}		[2]{\acrosymni		{#1.#2}{\acs {#1}\acs{#2}}{\acl{#2} \acl {#1}}}
\newcommand*{\acrodefcompni}	[2]{\acrodefsymni	{#1.#2}{\acs {#1}\acs{#2}}{\acl{#2} \acl {#1}}}

%\acro and \acrodef derivatives cumulative with \acrosymnp that creates a composite symbol and DOESN'T append "s" after the long form
\newcommand*{\acrocompnp}		[2]{\acrosymnp		{#1.#2}{\acs {#1}\acs{#2}}{\acl{#2} \acl {#1}}}
\newcommand*{\acrodefcompnp}	[2]{\acrodefsymnp	{#1.#2}{\acs {#1}\acs{#2}}{\acl{#2} \acl {#1}}}

%\acro and \acrodef derivatives cumulative with \acrosymnip that creates a composite symbol, WITHOUT the "at epoch i" variation and "s" after the long form
\newcommand*{\acrocompnip}		[2]{\acrosymnip		{#1.#2}{\acs {#1}\acs{#2}}{\acl{#2} \acl {#1}}}
\newcommand*{\acrodefcompnip}	[2]{\acrodefsymnip	{#1.#2}{\acs {#1}\acs{#2}}{\acl{#2} \acl {#1}}}

%% ======================== overloading 'xyz-component of the' forms of acronyms ========================== %%

%\acrodef derivatives cumulative with \acrodefsymplural that generate "x,y,z components of the #2 #1", using the long plural forms
\newcommand*{\acrodefcompvec}[2]{
  \acrodefsymplural  {#1.x.#2}  {\acs{#1}\acs{x}\acs{#2}}{\acl{x} of the \acl{#2} \acl {#1}}
                              {\acs{#1}\acs{x}\acs{#2}}{\acl{x} of the \acl{#2} \aclp{#1}}
  \acrodefsymplural  {#1.y.#2}  {\acs{#1}\acs{y}\acs{#2}}{\acl{y} of the \acl{#2} \acl {#1}}
                              {\acs{#1}\acs{y}\acs{#2}}{\acl{y} of the \acl{#2} \aclp{#1}}
  \acrodefsymplural  {#1.z.#2}  {\acs{#1}\acs{z}\acs{#2}}{\acl{z} of the \acl{#2} \acl {#1}}
                              {\acs{#1}\acs{z}\acs{#2}}{\acl{z} of the \acl{#2} \aclp{#1}}
}
%\acrodef derivatives cumulative with \acrodefsymplural that generate "x,y,z components of the #2 #1", NOT using the long plural forms
\newcommand*{\acrodefcompvecnp}[2]{
	\acrodefsymplural	{#1.x.#2}	{\acs{#1}\acs{x}\acs{#2}}{\acl{x} of the \acl{#2} \acl{#1}}
									{\acs{#1}\acs{x}\acs{#2}}{\acl{x} of the \acl{#2} \acl{#1}}
	\acrodefsymplural	{#1.y.#2}	{\acs{#1}\acs{y}\acs{#2}}{\acl{y} of the \acl{#2} \acl{#1}}
									{\acs{#1}\acs{y}\acs{#2}}{\acl{y} of the \acl{#2} \acl{#1}}
	\acrodefsymplural	{#1.z.#2}	{\acs{#1}\acs{z}\acs{#2}}{\acl{z} of the \acl{#2} \acl{#1}}
									{\acs{#1}\acs{z}\acs{#2}}{\acl{z} of the \acl{#2} \acl{#1}}
}
%\acrodef derivatives cumulative with \acrodefsymplural that generate "x,y,z components of the #1", using the long plural forms and superscripts
\newcommand*{\acrodefcompvecu}[1]{
  \acrodefsymplural {#1.xu} {\acs{#1}\acs{xu}}{\acl{x} of the \acl {#1}}
                            {\acs{#1}\acs{xu}}{\acl{x} of the \aclp{#1}}
  \acrodefsymplural {#1.yu} {\acs{#1}\acs{yu}}{\acl{y} of the \acl {#1}}
                            {\acs{#1}\acs{yu}}{\acl{y} of the \aclp{#1}}
  \acrodefsymplural {#1.zu} {\acs{#1}\acs{zu}}{\acl{z} of the \acl {#1}}
                            {\acs{#1}\acs{zu}}{\acl{z} of the \aclp{#1}}
}

%% ======================== olerloading composite forms of acronyms ========================== %%

%\acro and \acrodef derivatives cumulative with \acrosymnp that allow for composite symbols and the text between the two long forms (no plural forms)
\newcommand*{\acrocomptnp}[3]{
	\acrosymnp		{#1.#3}{\acs {#1}\acs{#3}}{\acl{#1}#2\acl {#3}}
}
\newcommand*{\acrodefcomptnp}[3]{
	\acrodefsymnp	{#1.#3}{\acs {#1}\acs{#3}}{\acl{#1}#2\acl {#3}}
}
%\acro and \acrodef derivatives cumulative with \acrosymnp that allow for double composite symbols and the texts between the three long forms (no plural forms)
\newcommand*{\acrocompttnp}[5]{
	\acrosymnp		{#1.#3.#5}{\acs{#1}\acs{#3}\acs{#5}}{\acl{#1}#2\acl{#3}#4\acl {#5}}
}
\newcommand*{\acrodefcompttnp}[5]{
	\acrodefsymnp	{#1.#3.#5}{\acs{#1}\acs{#3}\acs{#5}}{\acl{#1}#2\acl{#3}#4\acl {#5}}
}

%% ======================== fixing section title margins ========================== %%

\newcommand{\symbolsleftmarginfix}{-1.7cm}
\newcommand{\symbolsrightmarginfix}{0.0cm}

%% ======================== fixing section title margins ========================== %%

%don't use underscore '_' in the labels of the acronyms

\ifdefined\IncludeSymbolsText
The following convention distinguishes between scalar, vector and tensor quantities:
\begin{itemize*}
\item Scalar quantities are represented by lower-case or capital unformatted symbols, e.g.:
  \begin{itemize*}
  \item \ace[: ]{range};
  \item \ace[: ]{lat};
  \item \ace[: ]{grav.pot}.
\end{itemize*}
\item Vector quantities are represented by lower-case bold-face symbols, e.g.:
  \begin{itemize*}
  \item \ace[: ]{oe};
  \item \ace[: ]{grav.acc};
  \item \ace[: ]{uv}.
  \end{itemize*}
\item Matrix quantities are represented by capital bold-face symbols, e.g.:
  \begin{itemize*}
  \item \ace[: ]{fm.A};
  \item \ace[: ]{grav.grad}.
  \end{itemize*}
\end{itemize*}

The superscripts described in \ref{sec:acro_super} add an additional meaning to the original symbol, associated with the context in which it is defined, such as:
\begin{itemize*}
\item the \ace[, ]{orb.pos.for.1};
\item the \ace[, ]{grav.pot.ref};
\item the \ace[, ]{noise.v.ori.sat.2.pw}.
\end{itemize*}

Subscripts, like superscripts, also add a contextual meaning but are restricted to either a component of a vector, such as:
\begin{itemize*}
\item the \acl{x} of the \acl{grav.acc}, \acs{grav.acc.mag}\acs{x};
\item the value of the (time-varying) \ace[, ]{rc.i};
\item the \ace[ of \ace{degree} and \ace{order}, ]{sph.coef.lm}.
\end{itemize*}

Notice that it is possible to refer to the \acl{y} of the \acl{orb.vel} at \acl{epoch.i} with \acs{orb.vel}\acs{y}$_,$\acs{epoch.i} (which has the same meaning as \acs{orb.vel}\acs{epoch.i}$_,$\acs{y}) but if the number of superscripts is low, it is preferable to move the component of the vector\slash tensor to superscript, e.g. \acs{orb.vel}\acs{yu}\acs{epoch.i}. The subscripts used in the thesis are presented in \ref{sec:acro_sub}.

\pavel{\replace{We will assume that}{As a general rule,} all the quantities \replace{under consideration}{with index \acs{epoch.i.big}} are given with a constant sampling interval $\Delta\acs{t}$. A quantity with the \replace{lower index}{subscript} \replace{$i$}{\acs{epoch.i.big}} corresponds to the observation time \replace{$t_i := t_0+i\Delta t$, where $t_0$ is an initial epoch ($i=1,..N$ with $N$ being the total number of data)}{$\acs{t.i} := \acs{t}_0+\left(\acs{epoch.i.big}-1\right)\Delta\acs{t}$, where $\acs{t}_0$ is an initial epoch ($\acs{epoch.i.big}=1,..N$ with $N$ being the total number of data)}.\replace{ Sometimes, the lower index $i$ will be omitted in order to simplify the notation.}{}}

%moved from the modelling chapter intro
Regarding terminology, the term \emph{synthetic} is used as a means to express that a certain measurement is generated numerically on the basis of an assumed model, rather than being the output of a real sensor.

The electronic version of this document includes clickable links on most instances of symbols, superscripts, subscripts, pointing to their definition in \ref{sec:symbols}. The acronyms are also referenced to the corresponding entry of the list in \ref{sec:acronyms}.
\fi

\begin{acronym}[---------------]
%
\acrodef{error}[!error!]{!acronym error!}
%
\acrosymni      {a0}        {$A_0$}          {radial amplitude}
\acrosymni      {b0}        {$B_0$}          {cross-track amplitude}
\acrosymni      {alpha}      {$\alpha$}      {radial phase}
\acrosymni      {beta}      {$\beta$}        {horizontal phase}
\acrosymni      {xoff}      {$x_{\rm off}$}  {radial offset}
\acrosymni      {yoff}      {$y_{\rm off}$}  {along-track offset}

\acrodefcompni	{a0} 				{est}
\acrodefcompni	{b0} 				{est}
\acrodefcompni	{yoff}			{est}

\acrosymnip			{oe}				{${\bm{\epsilon}}$}			{orbital elements}
\acrodefsymnip	{mean.oe}		{$\bar{\acs{oe}}\acs{sat.1.and.2}$}
																										{mean \acl{oe} of \acl{sat.1.and.2}}
\acrodefsymnip	{rel.oe} 		{\acs{oe}\acs{sat.12}}	{relative \acl{oe} of \acl{sat.12}}
%don't use \ace, \act or \acf with the 3 acronyms below
\acrodefsymnip	{oe.j}			{\acs{oe}\acs{sat.j}}		{\acl{oe} of \acl{sat.j}}
\acrodefsymnip	{oe.1}			{\acs{oe}\acs{sat.1}}		{\acl{oe} of \acl{sat.1}}
\acrodefsymnip	{oe.2}			{\acs{oe}\acs{sat.2}}		{\acl{oe} of \acl{sat.2}}

\acrosymnip			{a}					{$a$}					{semi-major axis}
\acrosymnip			{e}					{$e$}					{eccentricity}
\acrosymnip			{i}					{$i$}					{inclination}
\acrosymnip			{raan}			{$\Omega$}		{right ascension of the ascending node}
\acrosymnip			{ap}				{$\omega$}		{argument of perigee}
\acrosymnip			{ma}				{$M$}					{mean anomaly}
\acrosymnip			{niu} 			{$\nu$}				{true anomaly}

\acrodefsymnip	{rel.a}			{\acs{a}\acs{sat.12}}			{\acl{a} of \acl{sat.12}}
\acrodefsymnip	{rel.e}			{\acs{e}\acs{sat.12}}			{\acl{e} of \acl{sat.12}}
\acrodefsymnip	{rel.i}			{\acs{i}\acs{sat.12}}			{\acl{i} of \acl{sat.12}}
\acrodefsymnip	{rel.raan}	{\acs{raan}\acs{sat.12}}	{\acl{raan} of \acl{sat.12}}
\acrodefsymnip	{rel.ap}		{\acs{ap}\acs{sat.12}}		{\acl{ap} of \acl{sat.12}}
\acrodefsymnip	{rel.ma}		{\acs{ma}\acs{sat.12}}		{\acl{ma} of \acl{sat.12}}
\acrodefsymnip	{rel.niu}		{\acs{niu}\acs{sat.12}}		{\acl{niu} \acl{sat.12}}

\acrosymni			{p.rev}			{$T$\acs{rev}}						{orbital \acl{rev} period}
\acrosymni			{p.revisit}	{$T$\acs{revisit}}				{\acl{revisit} period}

\newcommand*{\acromatrix}[3]{
	\acrosympluralni{#1}{#2}{#3 matrix}{#2}{#3 matrices}
}

\acrodefsymnip	{fm.y}				{${\bf y}$}							{data}
\acrosymni			{fm.m}				{${\bf m}$}							{model parameters}
\acrosymnip			{fm.m.j}			{${m_j}$}								{model parameter $j$}
\acrosymnip			{fm.m.res}		{\acs{fm.m}\acs{res}}		{model correction}
\acrosymni			{fm.phi} 			{${\bm{\Phi}}$}					{functional model}
\acrosymnip			{fm.phi.i}		{${\Phi_i}$}						{functional model element $i$}
\acromatrix			{fm.A}				{${\bf A}$}							{design}
\acrosymnip			{fm.A.ij}			{${A_{ij}}$}						{design matrix element $ij$}
\acromatrix			{fm.N}				{${\bf N}$}							{normal}
\acromatrix			{fm.Cd}				{${\bf C\acs{res}}$}		{data noise covariance}
\acromatrix			{fm.Cm0}			{${\bf C\acs{ref}}$}		{reference model noise covariance}
\acrocompnip		{fm.y}				{obs}
\acrocompnip		{fm.y}				{for}
\acrocompnip		{fm.y}				{res}
\acrocompnip		{fm.m}				{ref}

\acrosym				{lat}					{$\vartheta$}						{co-latitude}
\acrosym				{long}				{$\lambda$}							{longitude}
\acrosymplural	{rad}					{$r$}										{radius}
															{$r$}										{radii}
\acrosymnip			{rad.ref.ell}	{$R$}										{semi-major axis of a reference ellipsoid}
\acrosymnip			{G}						{$G_0$}									{universal constant of gravitation}
\acrosymnip			{Me}					{$M_{{\rm Earth}}$}			{mass of the Earth}

\acrosymnp	 		{t}						{$t$}										{time}
\acrosymnip			{delta.t}			{$\Delta t$}						{sampling rate}

\acrosymplural	{ang.vel}			{$\omega$}							{angular velocity}
															{$\omega$}							{angular velocities}
\acrosym				{ang.vel.vec}	{${\bm{\omega}}$}				{\acl{ang.vel} vector}
\acrosym				{ang.acc}			{$\dot\omega$}					{angular acceleration}
\acrosym				{ang.acc.vec}	{${\bm{\dot\omega}}$}		{\acl{ang.acc} vector}

\acrosymplural	{mass} 				{$m$}										{mass}
															{$m$}										{masses}
\acrosym				{force}				{${\bm{\zeta}}$}				{force}
\acrosympluralni{freq}				{$f$}										{frequency}
															{$f$}										{frequencies}

\acrosym				{avg.filt}		{${\bf w}$}							{averaging filter}

\acrosymni			{degree}			{$\ell$}								{degree}
\acrosymni			{degree.max}	{$L$\acs{max}}					{maximum degree}
\acrosymni			{order}				{$m$}										{order}
\acrosymni			{sph.harm}		{$\bar Y$}							{4$\pi$-normalised surface spherical harmonic function}
\acrosymni			{sph.harm.lm}	{$\bar Y$\acs{lm}}			{spherical harmonic function\acroextra{ of \ace{degree} and \ace{order}}}
\acrosym				{sph.coef}		{$\bar C$}							{Stokes coefficient}
\acrosymni			{sph.coef.lm}	{$\bar C$\acs{lm}}			{Stokes coefficient\acroextra{ of \ace{degree} and \ace{order}}}

\newcommand*{\acrostokes}[2]{
\acrodefsymnip	{#1.#2}				{\acs{#1}\acs{#2}}					{\aclp{#1} associated with the \acl{#2} gravity field}
\acrodefsymnip	{#1.#2.lm}		{\acs{#1}\acs{lm}\acs{#2}}	{\aclp{#1} associated with the \acl{#2} gravity field}
}
\acrostokes			{sph.coef}				{true}
\acrostokes			{sph.coef}				{ref}
\acrostokes			{sph.coef}				{st}
\acrostokes			{sph.coef}				{tv}
\acrodefsymnip	{sph.coef.tv.avg}	{$\overline{\acs{sph.coef.tv}}$}{\aclp{sph.coef} associated with the mean \acl{tv} gravity field}

%make sure this is in agreement with noise.gf
\acrosymni			{grav.pot}					{$V$}				{gravitational potential}
\acrodefcompni	{grav.pot}					{true}
\acrodefcompni	{grav.pot.true}			{est}
\acrodefcompni	{grav.pot}					{ref}
\acrodefcompni	{grav.pot}					{res}
\acrodefcompni	{grav.pot.res}			{est}
\acrodefcompni	{grav.pot}					{st}
\acrodefcompni	{grav.pot}					{tv}

\acrosym				{grav.acc.mag}			{$g$}				{gravitational acceleration magnitude}
\acrosym				{grav.acc}					{${\bf g}$}	{gravitational acceleration}
\acrodefcomp		{grav.acc}					{obs}
\acrodefcomp		{grav.acc}					{for}
\acrodefcomp		{grav.acc}					{res}
\acrosym        {grav.acc.avg}      {${\bf {\bar g}}$} {gravitational acceleration}
\acrodefcomp    {grav.acc.avg}      {obs}
\acrodefcomp    {grav.acc.avg}      {for}
\acrodefcomp    {grav.acc.avg}      {res}

\acrosymni			{grav.grad.comp}		{$G$}				{gravity gradient component}
\acrosymnip			{grav.grad.comp.ij}	{$G_{ij}$}	{gravity gradient component $ij$}
\acrosymni			{grav.grad}					{${\bf G}$}	{gravity gradient}
%if you change the G above, also change it in noise.sgg.G
\acrodefcompni	{grav.grad}					{obs}
\acrodefcompni	{grav.grad}					{for}

\acrosymni			{sgg.frame}			{\acs{grav.grad}\acs{frame}}	{\acl{frame} \acl{grav.grad}}
\acrosymni			{sgg.a}					{\acs{acc}\acs{frame}}				{\acl{frame} \acl{acc}}
\acrosymni			{sgg.f}					{${\bf f}$}										{electrostatic specific force}
\acrosymnip			{sgg.cent}			{${\bm{\Omega\Omega}}$}				{\acl{grav.grad} associated with centrifugal accelerations}
\acrosymnip			{sgg.euler}			{${\bm{\dot\Omega}}$}					{\acl{grav.grad} associated with Euler accelerations}
\acrosympluralni{sgg.r}					{${\bf r}$}										{position of the proof mass relative to the \acs{COM} of the satellite}
																{${\bf r}$}										{positions of the proof mass relative to the \acs{COM} of the satellite}
\acrosympluralni{sgg.rdot}			{${\bf \dot r}$}							{velocity of the proof mass relative to the \acs{COM} of the satellite}
																{${\bf \dot r}$}							{velocities of the proof mass relative to the \acs{COM} of the satellite}
\acrosympluralni{sgg.rddot}			{${\bf \ddot r}$}							{acceleration of the proof mass relative to the \acs{COM} of the satellite}
																{${\bf \ddot r}$}							{accelerations of the proof mass relative to the \acs{COM} of the satellite}
\acrosymnip			{sgg.rddot.crf}	{\acs{sgg.rddot}\acs{c.rf}}		{inertial acceleration of the proof mass}

\newcommand*{\acrosggf}[1]{
	\acrodefsymnip{sgg.f.#1}{\acs{sgg.f}\acs{acc.#1}}{\acl{sgg.f} acting on the proof mass inside \acl{acc.#1}}
}
\acrosggf{1}\acrosggf{2}\acrosggf{3}\acrosggf{4}\acrosggf{5}\acrosggf{6}

\newcommand*{\acrosggr}[1]{
	\acrodefsymnip{sgg.r.#1}{\acs{sgg.r}\acs{acc.#1}}{position of the proof mass of the \acl{acc.#1} relative to the \acs{COM} of the satellite}
}
\acrosggr{1}\acrosggr{2}\acrosggr{3}\acrosggr{4}\acrosggr{5}\acrosggr{6}\acrosggr{p}

%\acro{grav.field}[$\mathbb{G}$]{gravitational field}
%\acrodef{grav.field.true}[\acs{grav.field}\acs{true}]{\acl{true} \acl{grav.field}}
%\acrodefplural{grav.field.true}[\acs{error}]{\acl{error}}
%\acrodef{grav.field.ref}[\acs{grav.field}\acs{ref}]{\acl{ref} \acl{grav.field}}
%\acrodefplural{grav.field.ref}[\acs{error}]{\acl{error}}

\acrosym        {acc}          {${\bf a}$}                          {acceleration}
\acrosym        {acc.mag}      {$a$}                                {\acl{acc} magnitude}
\acrodefcompvecu{acc.mag}
\acrosym        {acc.avg}      {${\bf \bar a}$}                    {averaged \acl{acc}}
\acrocomp        {acc}          {ng}
\acrocomp        {acc}          {fr}
\acrocomp        {acc}          {cor}
\acrocomp        {acc}          {eul}
\acrocomp        {acc}          {is}

\acrosym				{isa}					{\acs{acc}\acs{is}}									{point-wise \acl{is} \acl{acc}}
\acrosymnp			{isa.x}				{\acs{isa}\acs{los.rf}\acs{x}}			{\acl{x} of the \acl{isa} in the \acl{los.rf}}
\acrosym				{isa.ng}			{\acs{acc}\acs{is}\acs{ng}}					{point-wise \acl{is} \acl{ng} \acl{acc}}
\acrosymnp			{isa.los}			{\acs{acc.mag}\acs{is}\acs{los.v}}	{\acl{isa} projected into the \acl{los.v}}

\acrosym				{uv}					{${\mathsf {\bf e}}$}								{unit vector}
\acrosymnip			{uv.los}			{\acs{uv}\acs{los.v}}								{unit vector defining the \acl{los.v}}
\acrosymnip			{uv.los.i}		{\acs{uv}\acs{los.v}\acs{epoch.i}}	{unit vectors defining the \acl{los.v}}
\acrocomp				{uv.los}			{for}
\acrocomp				{uv.los}			{obs}
\acrosym				{v.los}				{${\bf d}$}													{\ac{LOS} vector}

\acrosymnip      {theta}        {$\theta$}%
                              {angle between the \ac{LOS} vectors at successive epochs}
\acrodefsymnip  {theta.i}      {\acs{theta}\acs{epoch.i}}%
                              {angles \acs{theta} between the \ac{LOS} vectors at successive epochs}
%don't use \ace, \act or \acf with these two acronyms
\acrodefsymnip	{theta.minus}	{\acs{theta}\acs{epoch.i}$_{-}$}%
															{angle between the \ac{LOS} vector at \acl{epoch.i} and at the previous epoch}
\acrodefsymnip	{theta.plus}	{\acs{theta}\acs{epoch.i}$_{+}$}%
															{angle between the \ac{LOS} vector at \acl{epoch.i} and at the following epoch}

%change \acro{noise.rho} if you change this
\acrosym				{range}				{$\rho$}			{range}
\acrocomp				{range}				{for}
\acrocomp				{range}				{obs}
\acrocomp				{range}				{est}
\acrocomp				{range}				{max}
\acrocomp				{range}				{min}
\acrocomp				{range}				{avg}
\acrosym				{range-rate}	{$\dot \rho$}	{range-rate}

\acrosym				{rc}					{$\bar a$}		{range combination}
\acrocomp				{rc}					{ref}
\acrocomp				{rc}					{obs}
\acrocomp				{rc}					{res}

\newcommand*{\acrosats}[1]{
	\acrosymplural		{#1.1}		{\acs{#1}\acs{sat.1}}			{\acl{#1} of \acl{sat.1}}
															{\acs{#1}\acs{sat.1}}			{\aclp{#1} of \acl{sat.1}}
	\acrosymplural		{#1.2}		{\acs{#1}\acs{sat.2}}			{\acl{#1} of \acl{sat.2}}
															{\acs{#1}\acs{sat.2}}			{\aclp{#1} of \acl{sat.2}}
	\acrosymplural		{#1.j}		{\acs{#1}\acs{sat.j}}			{\acl{#1} of \acl{sat.j}}
															{\acs{#1}\acs{sat.j}}			{\aclp{#1} of \acl{sat.j}}
	\acrosymplural		{#1.12}		{\acs{#1}\acs{sat.12}}		{\acl{#1} of \acl{sat.12}}
															{\acs{#1}\acs{sat.12}}		{\aclp{#1} of \acl{sat.12}}
	\acrosymplural		{#1.m12}	{\acs{#1}\acs{sat.m12}}		{\acl{#1} of the \acl{sat.m12}}
															{\acs{#1}\acs{sat.m12}}		{\aclp{#1} of the \acl{sat.m12}}
}

%make sure this is in agreement with rpos
\acrosym			{orb.pos.comp}		{$x$}											{orbital position\acroextra{ component}}
\acrosym			{orb.pos}					{${\bf x}$}								{orbital position}
\acrosats			{orb.pos}
\acrosats			{orb.pos.comp}
\acrocomp			{orb.pos}					{obs}
\acrosats			{orb.pos.obs}
\acrocomp			{orb.pos.obs}			{adj}
\acrosats			{orb.pos.obs.adj}
\acrocomp			{orb.pos}					{for}
\acrosats			{orb.pos.for}

\acrosym			{orb.pos.kpl}			{\acs{orb.pos}\acs{kpl}}	{\acl{kpl} reference orbit}
\acrosym			{orb.pos.sim}			{\acs{orb.pos}\acs{sim}}	{\acl{sim} orbit}
\acrosym			{orb.pos.Sim}			{\acs{orb.pos}\acs{Sim}}	{\acl{Sim} orbit}

%make sure this is in agreement with rvel
\acrosym			{orb.vel.comp}		{$\dot x$}								{orbital velocity\acroextra{ component}}
\acrosymplural{orb.vel}					{${\bf \dot x}$}					{orbital velocity}
																{${\bf \dot x}$}					{orbital velocities}
\acrosats			{orb.vel}

\acrodefsymnp	{orb.vel.12.z}		{\acs{orb.vel.comp}\acs{sat.12}\acs{z}}
																													{\acl{z} of the \acl{orb.vel} of \acl{sat.12}}


\acrosymnp		{noise}						{$\delta$}								{noise\acroextra{ (scalar)}}
\acrosymnp		{noise.v}					{${\bm{\delta}}$}					{\acl{noise}\acroextra{ (vector)}}
\acrosymnp		{noise.t}					{${\bm{\Delta}}$}					{\acl{noise}\acroextra{ (tensor)}}

%make sure this is in agreement with grav.pot
\acrodefsymplural  {noise.gf}    {${\bm\Delta}^{\rm\left(V\right)}$}        {gravity field error}
                                {${\bm\Delta}^{\rm\left(V\right)}$}        {error in the gravity field models}
\acrodefsymnp  {noise.res3rc}    {${\bm\Delta}^{\rm\left(V,GRACE\right)}$}  {\acs{GRACE} a posteriori residuals}


\acrodefsymnp	{Noise}						{\acs{noise}}							{Noise}
\acrodefsym		{noise.comp}			{\acs{noise}\acs{comp.k}}	{\acl{noise} \acl{comp.k}}
\acrodefsym		{Noise.comp}			{\acs{Noise}\acs{comp.k}}	{\acl{Noise} \acl{comp.k}}

\acrosymnp		{noise.obs}				{\acs{noise}\acs{obs}}		{observation \acl{noise}} 	%no \acrocomp here
\acrodefsymnp	{noise.v.obs}			{\acs{noise.v}\acs{obs}}	{observation \acl{noise.v}}	%no \acrocomp here
\acrosymnp		{noise.for}				{\acs{noise}\acs{for}}		{forecast \acl{noise}}		%no \acrocomp here
\acrodefsymnp	{noise.v.for}			{\acs{noise.v}\acs{for}}	{forecast \acl{noise.v}}	%no \acrocomp here

%hlsst observation noise
\acrocompnp		{noise.obs}				{hlsstu}

%relative positioning noise
\acrocompnp		{noise}						{rpos}
\acrocompnp		{noise.v}					{rpos}
\acrocompnp		{noise}						{rvel}
\acrocompnp		{noise.v}					{rvel}

\acrodefcompvecnp{noise}				{rpos}
\acrodefcompvecnp{noise}				{rvel}

%orbital noise
\acrocomptnp	{noise}						{ in the }		{orb.posu}
\acrocomptnp	{noise.v}					{ in the }		{orb.posu}
\acrocomptnp	{noise.v}					{ in the }		{orb.pos.ju}
\acrocomptnp	{noise.v}					{ in the }		{orb.pos.1u}
\acrocomptnp	{noise.v}					{ in the }		{orb.pos.2u}
\acrocompttnp	{noise}						{ in the }		{x}				{ of the }		{orb.posu}
\acrocompttnp	{noise}						{ in the }		{y}				{ of the }		{orb.posu}
\acrocompttnp	{noise}						{ in the }		{z}				{ of the }		{orb.posu}

%change this if you change \acro{range}
\acrosymnp		{noise.rho}			{\acs{noise}$^{\rm {\left(\rho\right)}}$}	{ranging sensor \acl{noise}}

\acrocomp			{noise.v}					{accu}
\acrodefcomp	{noise.v.accu}		{pw}
\acrocompttnp	{noise.v}					{ of the } 		{accu}		{ of the } 		{sat.1}
\acrocompttnp	{noise.v}					{ of the } 		{accu}		{ of the } 		{sat.2}
\acrocompttnp	{noise.v}					{ of the } 		{accu}		{ of the } 		{sat.j}

\acrocomp			{noise}						{ran}
\acrodefcomp	{noise}						{Ran}
\acrocomp			{noise}						{ran.spl}
\acrocomp			{noise}						{rpos.rc}
\acrodefcomp	{noise}						{Rpos.rc}
\acrodefcomp	{noise}						{rpos.rc.a}
\acrocomp			{noise}						{apos.rc}
\acrodefcomp	{noise}						{Apos.rc}
\acrocomp			{noise}						{cor}
\acrodefcomp	{noise}						{Cor}
\acrocomp			{noise}						{ng}
\acrodefcomp	{noise}						{Ng}
\acrodefcomp	{noise.ng}				{pw}

\acrocomp			{noise}						{pos}
\acrodefcomp	{noise}						{Pos}
\acrocomp			{noise.v}					{pos}
\acrocomp			{noise.v.pos}			{pw}
\acrocomptnp	{noise.pos}				{ of }			{sat.1}
\acrocomptnp	{noise.pos}				{ of }			{sat.2}
\acrocomptnp	{noise.pos}				{ of }			{sat.j}

\acrocomp			{noise}						{ori}
\acrodefcomp	{noise}						{Ori}
\acrodefcomp	{noise.v}					{ori}
\acrodefcomp	{noise.v.ori}			{pw}
\acrocomptnp	{noise.v.ori}			{ of }			{sat.1}
\acrocomptnp	{noise.v.ori}			{ of }			{sat.2}
\acrocomptnp	{noise.v.ori}			{ of }			{sat.j}
\acrodefcomp	{noise.v.ori.sat.1}	{pw}
\acrodefcomp	{noise.v.ori.sat.2}	{pw}
\acrodefcomp	{noise.v.ori.sat.j}	{pw}

\acrosymnp		{noise.orb.vel.12.z}
							{\acs{noise}$^{\left(\mbox{\scriptsize{\acs{orb.vel.12.z}}}\right)}$}	{\acl{noise} in the \acl{orb.vel.12.z}}
\acrosymnp		{noise.uv.los}
							{\acs{noise.t}\acs{los.v}}	{tensor describing the \acl{noise.t} in the orientation of the \acs{LOSRF}}

\acrosymnp		{noise.st}				{\acs{noise}\acs{st}}		{residual \acl{st} signal}
\acrodef			{noise.St}				[\acs{noise.st}]				{Residual \acl{st} signal}
\acrosymnp		{noise.tv}				{\acs{noise}\acs{tv}}		{residual \acl{tv} signal}
\acrodef			{noise.Tv}				[\acs{noise.tv}]				{Residual \acl{tv} signal}
\acrosymnp		{noise.sp}				{\acs{noise}\acs{sp}}		{omission signal}
\acrodef			{noise.Sp}				[\acs{noise.sp}]				{Omission signal}

\acrosymnp		{noise.sgg.r}			{\acs{noise.v}$^{\rm\left(r\right)}$}
																												{proof-mass positioning \acl{noise.v}}
\acrosymnp		{noise.sgg.f}			{\acs{noise.v}$^{\rm\left(f\right)}$}
																												{\acl{grav.grad} sensor \acl{noise.t}}
\acrosymnp		{noise.sgg.O}			{\acs{noise.t}$^{\rm\left({\bm{\Omega}}\right)}$}
																												{\acl{grav.grad} orientation \acl{noise.t}}
\acrosymnp		{noise.sgg.G}			{\acs{noise.t}$^{\rm\left({\bf G}\right)}$}
																												{total \acl{grav.grad} measurement \acl{noise.t}}



\ifdefined\IncludeSymbolsText
\begin{changemargin}{\symbolsleftmarginfix}{\symbolsrightmarginfix}
%----------------------------------------------------------
%----------------------------------------------------------
\ifdefined\IncludeSymbolsSections
\section{Mathematical operations}
\label{sec:acro_math}
\else
\medskip{\large \bf Mathematical operations}\medskip
\fi
%----------------------------------------------------------
%----------------------------------------------------------
\end{changemargin}
\fi

\acro					{gradient}				[${\bm{\nabla}}$]				{gradient}
\acro					{double.gradient}	[${\bm{\nabla\nabla}}$]	{double gradient}
\acro					{Laplace}					[$\nabla^2$]						{Laplace}
\acro					{conv}						[$\ast$]								{convolution}
\acro					{transpose}				[$^{\mathrm{T}}$]				{transpose}
% \acro					{transpose}				[$^\intercal$]					{transpose}
\acro					{fourier}					[$\mathscr{F}$]					{Fourier transform}
\acro					{norm}						[\norm{\bm{v}}]					{norm (or length) of vector ${\bm{v}}$} %\acused{norm}

\ifdefined\IncludeSymbolsText
\begin{changemargin}{\symbolsleftmarginfix}{\symbolsrightmarginfix}

\[
\acs{gradient} = \left[\dfrac{{d}}{{dx_1}}, ..., \dfrac{{d}}{{dx_n}}\right]^\mathrm{T}
\]

\[
\acs{double.gradient} = \acs{gradient} \cdot \acs{gradient}\acs{transpose}
= \left( {\begin{array}{*{20}c}
	{\frac{{d^2}}{{dx_1 ^2 }}} & \cdots & {\frac{{d^2}}{{dx_1 dx_n }}} \\
	\vdots & \ddots & {} \\
	{\frac{{d^2}}{{dx_n dx_1 }}} & {} & {\frac{{d^2}}{{dx_n ^2 }}} \\
\end{array}} \right)
\]

\[
\acs{Laplace} = \acs{gradient}\acs{transpose} \cdot \acs{gradient} = \frac{{d^2}}{{dx_1 ^2 }} + ... + \frac{{d^2}}{{dx_n ^2 }}
\]
\end{changemargin}
\fi

\ifdefined\IncludeSymbolsText
\begin{changemargin}{\symbolsleftmarginfix}{\symbolsrightmarginfix}
%----------------------------------------------------------
%----------------------------------------------------------
\ifdefined\IncludeSymbolsSections
\section{Superscripts}
\label{sec:acro_super}
\else
\medskip{\large \bf Superscripts}\medskip
\fi
%----------------------------------------------------------
%----------------------------------------------------------

Superscripts add an additional meaning to the original symbol, associated with the context in which it is defined, such as the \ace{orb.pos.for.1}, the \ace{grav.pot.ref}, or the \acl{orb.pos} defined in the \acl{los.rf} \acl{orb.pos} \acs{orb.pos}\acs{los.rf}.

\end{changemargin}
\fi

\newcommand*{\acrosuper}[3]{
	\acro{#1}[$^{\rm{\left(#2\right)}}$]{#3}
}

\acro				{est}					[$^\ast$]				{estimated}
\acrosuper	{cent}				{cent}					{centrifugal}
\acrosuper	{noisy}				{\acs{noise}}		{noisy}
\acrosuper	{fr}					{fr}						{frame rotation}
\acrosuper	{cor}					{Cor}						{Coriolis}
\acrosuper	{eul}					{Eul}						{Euler}

\acrosuper	{ref}					{ref}						{reference}
\acrosuper	{true}				{true}					{\quotes{true}}
\acrosuper	{for}					{for}						{forecasted}
\acrosuper	{obs}					{obs}						{observed}
\acrosuper	{res}					{res}						{residual}

\acrosuper	{adj}					{adj}						{adjusted}
\acrosuper	{pw}					{pw}						{point-wise}

\acrosuper	{comp.k}			{k}							{\acroextra{\acl{noise} }type}
\acrosuper	{gf}					{\acs{sph.coef}}{gravity field model}

\acrosuper	{sat.1}				{1}							{satellite 1}
\acrosuper	{sat.2}				{2}							{satellite 2}
\acrosuper	{sat.12}			{12}						{satellite 1 relatively to satellite 2}
\acrosuper	{sat.1.and.2}	{1,2}						{satellite 1 and satellite 2}
\acrosuper	{sat.m12}			{\overline{12}}	{middle-point between \acl{sat.1} and \acl{sat.2}}
\acrodefplural{sat.m12}		[\acs{sat.m12}]	{middle-points between \acl{sat.1} and \acl{sat.2}}
%make sure this symbol is consistent below
\acrodef		{sat.j.big}		[$j$]						{\acl{error}}
\acrosuper	{sat.j}				{j}							{satellite \acs{sat.j.big}}

\acrosuper	{accu}				{acc}						{accelerometer}
\acrosuper	{acc.1}				{1}							{accelerometer 1}
\acrosuper	{acc.2}				{2}							{accelerometer 2}
\acrosuper	{acc.3}				{3}							{accelerometer 3}
\acrosuper	{acc.4}				{4}							{accelerometer 4}
\acrosuper	{acc.5}				{5}							{accelerometer 5}
\acrosuper	{acc.6}				{6}							{accelerometer 6}
\acrosuper	{acc.12}			{12}						{accelerometer 1 relative to accelerometer 2}
%make sure this symbol is consistent below
\acrodef		{acc.p.big}		[$p$]						{\acl{error}}
\acrosuper	{acc.p}				{p}							{accelerometer \acs{acc.p.big}}

%if you change any of these, do it also above in the acronyms section
\acrosuper	{c.rf}				{CRF}						{\acl{CRF}}
\acrosuper	{t.rf}				{TRF}						{\acl{TRF}}
\acrosuper	{lo.rf}				{LORF}					{\acl{LORF}}
\acrosuper	{lh.rf}				{LHRF}					{\acl{LHRF}}
\acrosuper	{los.rf}			{LOSRF}					{\acl{LOSRF}}
\acrosuper	{s.rf}				{SRF}						{\acl{SRF}}
\acrosuper	{g.rf}				{GRF}						{\acl{GRF}}
\acrosuper	{h.rf}				{HRF}						{\acl{HRF}}

\acrosuper	{ran}					{R}							{ranging}
\acrodef		{Ran}					[\acs{ran}]			{Ranging}
\acrosuper	{ran.spl}			{R,spl}					{simplistic ranging}
\acrosuper	{rpos.rc}			{rP}						{relative position}
\acrodef		{Rpos.rc}			[\acs{rpos.rc}]	{Relative position}
\acrosuper	{rpos.rc.a}		{rP,alt}				{alternative relative position}
\acrosuper	{apos.rc}			{aP}						{absolute position}
\acrodef		{Apos.rc}			[\acs{apos.rc}]	{Absolute position}
\acrosuper	{ng}					{acc}						{accelerometer}
\acrodef		{Ng}					[\acs{acc}]			{Accelerometer}
\acrosuper	{pos}					{P}							{positioning}
\acrodef		{Pos}					[\acs{pos}]			{Positioning}
\acrosuper	{ori}					{L}							{orientation}
\acrodef		{Ori}					[\acs{ori}]			{Orientation}
\acrosuper	{cor}					{C}							{correction}
\acrodef		{Cor}					[\acs{cor}]			{Correction}

\acrosuper	{dlo}					{DA60}					{amplitude at degree 60}
\acrosuper	{dup}					{DA100}					{amplitude at degree 100}
\acrosuper	{dsu}					{CDA120}				{cumulative amplitude at degree 120}


\acrosuper	{st}					{st}						{static}
\acrosuper	{tv}					{tv}						{time-variable}
\acrosuper	{sp}					{sp}						{omission error}

%make sure these symbols is consistent below
\acrosuper	{is}					{12}				    {inter-satellite}
\acrosuper	{orb.posu}		{{\bf x}}				{\acl{orb.pos}}
\acrosuper	{orb.pos.ju}	{{\bf x}^{(j)}}	{\acl{orb.pos.j}}
\acrosuper	{orb.pos.1u}	{{\bf x}^{(1)}} {\acl{orb.pos.1}}
\acrosuper	{orb.pos.2u}	{{\bf x}^{(2)}} {\acl{orb.pos.2}}
\acrosuper	{rpos}				{\Delta {\bf x}}
                                          {relative orbit position}
\acrosuper	{rvel}				{\Delta {\bf \dot x}}
																					{relative orbit velocity}
\acrosuper	{los.v}				{LOS}						{\ac{LOS} direction}

\acrosuper	{xu}					{x}							{$x$-component}
\acrosuper	{yu}					{y}							{$y$-component}
\acrosuper	{zu}					{z}							{$z$-component}
\acrosuper	{xxu}					{xx}						{$xx$-component}
\acrosuper	{yyu}					{yy}						{$yy$-component}
\acrosuper	{zzu}					{zz}						{$zz$-component}

\acrosuper	{max}					{max}						{maximum}
\acrosuper	{min}					{min}						{minimum}
\acrosuper	{avg}					{avg}						{average}
\acrosuper	{frame}				{frame}					{frame}
\acrosuper	{kpl}					{kpl}						{Keplerian}
\acrosuper	{sim}					{sim}						{simulated}
\acrodef		{Sim}					[\acs{sim}]			{Simulated}

\acrosuper	{hlsstu}			{hl-SST} 				{\acl{hlsst}}

\acrosuper	{rev}					{rev} 					{revolution}
\acrosuper  {revisit}			{revisit}				{revisit}

\ifdefined\IncludeSymbolsText
\begin{changemargin}{\symbolsleftmarginfix}{\symbolsrightmarginfix}
%----------------------------------------------------------
%----------------------------------------------------------
\ifdefined\IncludeSymbolsSections
\section{Subscript}
\label{sec:acro_sub}
\else
\medskip{\large \bf Subscript}\medskip
\fi
%----------------------------------------------------------
%----------------------------------------------------------

Subscripts, like superscripts, also add a contextual meaning but are restricted to either a component of a vector, such as the \acl{x} of the \acl{grav.acc} \acs{grav.acc.mag}\acs{x}, or the value of the (time-varying) \acl{rc} at \acl{epoch.i}, \acs{rc}\acs{epoch.i}. Notice that it is possible to refer to the \acl{y} of the \acl{orb.vel} at \acl{epoch.i} with \acs{orb.vel}\acs{y}$_,$\acs{epoch.i} (which has the same meaning as \acs{orb.vel}\acs{epoch.i}$_,$\acs{y}) but if the number of superscripts is low, it is preferable to move the component of the vector\slash tensor to superscript, e.g. \acs{orb.vel}\acs{yu}\acs{epoch.i}.

\end{changemargin}
\fi

%make sure this symbol is consistent below
\acrodef			{epoch.i.big}	[$i$]						{\acl{error}}
\acro					{epoch.i}			[$_{i}$]				{epoch \acs{epoch.i.big}}
\acrodefplural{epoch.i}			[\acs{epoch.i}]	{epochs \acs{epoch.i.big}}
\acro					{x}						[$_x$]					{$x$-component}
\acro					{y}						[$_y$]					{$y$-component}
\acro					{z}						[$_z$]					{$z$-component}
\acro					{xx}					[$_{xx}$]				{$xx$-component}
\acro					{xy}					[$_{xy}$]				{$xy$-component}
\acro					{xz}					[$_{xz}$]				{$xz$-component}
\acro					{yy}					[$_{yy}$]				{$yy$-component}
\acro					{yx}					[$_{yx}$]				{$yx$-component}
\acro					{yz}					[$_{yz}$]				{$yz$-component}
\acro					{zz}					[$_{zz}$]				{$zz$-component}
\acro					{zx}					[$_{zx}$]				{$zx$-component}
\acro					{zy}					[$_{zy}$]				{$zy$-component}
\acro					{lm}					[$_{\ell m}$]		{degree and order}

%----------------------------------------------------------
%----------------------------------------------------------

\end{acronym}


